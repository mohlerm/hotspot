This chapter describes the implementation of the cached profiles implementation for Hotspot, written as part of this thesis.
\\\\
Most of the work is included in two new classes \texttt{/share/vm/ci/ciCacheProfiles.cpp} and \\\texttt{/share/vm/ci/ciCacheProfilesBroker.cpp} as well as modifications to \texttt{/share/vm/ci/ciEnv.cpp} and \texttt{/share/vm/compiler/compileBroker.cpp}. A full list of modified files and the changes can be seen in the webrev or appendix TODO.
\\\\
The changes are provided in form of a patch for Hotspot version 8182 TODO. This original version is referred to as \textit{Baseline}.
\\\\
I will describe and explain the functionality and the implementation design decision in the following sections, ordered by the appearance in execution.


\section{Creating Profiles}
The baseline version of Hotspot already offered a functionality to replay a compilation based on dumped profiling information.
This is mainly used in case the JVM crashes during JIT compilation to replay the compilation again and help finding the cause of this crash.
Dumping the data needed for the replay is either be done automatically in case of a crash or can be invoked manually by specifying the \texttt{DumpReplay} compile command option per method.
I introduce method option called \texttt{DumpProfile} as well as a compiler flag \texttt{-XX:+DumpProfiles} that appends profiling information to a file as soon as the method gets compiled. The first option can be specified as part of the \texttt{-XX:CompileCommand} or \texttt{-XX:CompileCommandFile} flag
